\documentclass[onecolumn,amsmath,amssymb,nofootinbib,prl]{revtex4-1}


\usepackage{comment}
\usepackage{graphicx} 
\usepackage{xcolor}
\usepackage{bm}

\usepackage[%frenchlinks=true,
colorlinks=true,
    linkcolor={red!50!black},
    citecolor={blue!50!black},
    urlcolor={blue!80!black}]{hyperref}


\begin{document}

\title{Engineering momentum profiles of cold-atom beams}

\author{D~Hudson \surname{Smith}}
\affiliation{Clemson University, Clemson, South Carolina 29634, USA}

\author{Artem~G \surname{Volosniev}}
\affiliation{Institut f{\"u}r Kernphysik, Technische Universit{\"a}t Darmstadt, 64289 Darmstadt, Germany}



\begin{abstract}
We ...
\end{abstract}


\maketitle

\section{Introduction}

Scattering is one of the most important observational tools in science. In physics, it is used in almost all subfields to understand internal structure of particles, materials, structures (cite).  

In systems of cold atoms scattering might seem obsolete because the energies are so low that usually only a few parameters determine physics, i.e., the system is in a universal regime. However, even at these low temperatures observables can strongly depend on the incoming momentum (see Figure 1a). For example, ...

{\it Figure 1a: examples where observables depend strongly on the incoming momentum: time-dependent one- and two-body scattering, three-body recombination,  polaron. 
Figure 1b: a sketch of the set-up for engineering beams with desirable momentum profiles: two reservoirs and a link between them.}

Here we study how to create fluxes with cold atoms, which can be used to study systems of cold atoms in scattering experiments. We illustrate using the one-dimensional Bose polaron problem.

\section{Design}

\subsection{Physical System}
Describe the system. The system has three parts: Two reservoirs and a link between them; see Figure 1a. One reservoir is made of a non-interacting (for simplicity) gas of particles at temperature $T$. We call these particles probes or $A$. The reservoir is connected to the link. The link is a potential that acts only on particles $A$ (for simplicity). The second reservoir is made of interacting particles $B$. These particles make a system of interest, which we would like to analyze. 

\subsection{Determining the Link Potential}
We now present a procedure for determining a link potential that produces a generic desirable flux properties. This simple procedure performs a global search over a family of possible link potentials in order to minimize the difference between the desired flux properties and the actual flux properties. Due to the non-convex nature of this optimization problem, we perform this search using the well-known global optimization routine called Differential Evolution (DE) \cite{original DE paper}. Despite its astounding simplicity, this algorithm does an excellent job of balancing the need to search the entire space against the need to learn from subsequent trials in the optimization process in order to speed up convergence. 

To start, we must specify the ideal desired flux profile $T(k)$ where $k$ is the incident momentum. This could be, for instance, a ``comb'' profile that only allows particle flux near a specified momentum, but our prescription applies to a broad class of flux profiles.

We also must specify the form of the link potential, $V(x, \bm{\theta})$, with unknown parameters $\bm{\theta}$. The functional form of the link potential should be chosen to represent the class of physically realizable potentials within the experimental context. In our case, we assume a sum of Gaussian potentials:
\begin{equation}
V(x,\{\bm{A},\bm{\mu},\bm{\sigma} \}) = \sum_{i=1}^{N}\frac{A_i}{\sqrt{2\pi\sigma_i^2}}\exp\left[{-\frac{(x-\mu_i)^2}{2\sigma_i^2}}\right],
\end{equation}
where $N$ is the number of Gaussian potentials (and the length of the parameter vectors $\bm{A}$, $\bm{\mu}$, and $\bm{\sigma}$).

In addition to specifying the form of the link potential, we may also need to specify a set of constraints on the parameters themselves in order to reflect the limitations in creating the physical potentials. In our case, we will enforce the following constraints
\begin{align}
  \sum_{i=1}^{N}\mu_i &= 0 & \\
  \sigma_i &> 0 &\forall i \in\{1,2,\ldots,N\}\\
  0 &\leq A_i < A^\mathrm{max} &\forall i \in\{1,2,\ldots,N\}             
\end{align}

In addition there may be generic constraints on the potential itself 

 It is clear that by changing the potential in the link we can change the flux. Let us describe how we compute potentials to create desirable fluxes. In this section we discuss this.  Describe the procedure here: We set the desirable momentum profile, guess a potential, calculate the corresponding transmission coefficient, minimize the parameters. 

{\it Figure 2: Examples of a few profiles that can be created: comb, evaporative, anti-evaporative etc. In the insets are the corresponding potentials.}

\section{Illustration}

The Bose polaron problem is very interesting, and many groups are studying it now. It is an open question how to measure the effective mass of the impurity in one-dimensional set-ups. There was one experiment, however the results are unclear. Also it is interesting to understand when the model breaks, i.e., at which momenta. We propose the following experiment: One reservoir -- probes, the link is from Fig. 2 (comb), the second reservoir is an interacting homogeneous Bose gas (for simplicity). Provide some realistic numbers. Probably some simple calculations (in the suppl. material). 

{\it Figure 3: Bose polaron set-up. Measurement of the effective mass -- the change of the energy upon interaction change.}


\section{Summary} 

We have shown how to create beams of particles to probe systems of cold atoms. We have illustrated the idea using the Bose polaron problem. Other interesting examples include ... 



\bibliographystyle{apsrev4-1}
\bibliography{bib}

 \end{document}


