%%%%%%%%%%%%%%%%%%%%%%% file template.tex %%%%%%%%%%%%%%%%%%%%%%%%%
%
% This is a template file for the LaTeX package SVJour2 for the
% Springer journal
%  The 20th European Conference on Few-Body Problems in Physics,
%  9-13 September 2013, Cracow, Poland
%
% Copy it to a new file with a new name and use it as the basis
% for your article. Delete % as needed.
%
%%%%%%%%%%%%%%%%%%%%%%%%%%%%%%%%%%%%%%%%%%%%%%%%%%%%%%%%%%%%%%%%%%%
%
% First comes an example EPS file -- just ignore it and
% proceed on the \documentclass line
% your LaTeX will extract the file if required
\begin{filecontents*}{example.eps}
%!PS-Adobe-3.0 EPSF-3.0
%%BoundingBox: 19 19 221 221
%%CreationDate: Mon Sep 29 1997
%%Creator: programmed by hand (JK)
%%EndComments
gsave
newpath
  20 20 moveto
  20 220 lineto
  220 220 lineto
  220 20 lineto
closepath
2 setlinewidth
gsave
  .4 setgray fill
grestore
stroke
grestore
\end{filecontents*}
%
\documentclass[onecolumn, 12pt, draft]{revtex4-1}
\bibpunct{[}{]}{;}{n}{}{,} % to get "[numbered]" references from natbib
%
\smartqed  % flush right qed marks, e.g. at end of proof
%
\usepackage{graphicx}
%
% \usepackage{mathptmx}      % use Times fonts if available on your TeX system
%
% insert here the call for the packages your document requires
\usepackage{amsmath, amssymb}
\usepackage{color}
% etc.
%
% please place your own definitions here and don't use \def but
% \newcommand{}{}
%
\journalname{Few-Body Systems}
%

\begin{document}

\title{Impenetrability in Floquet scattering in one dimension}
\author{A.~G.~Volosniev \and D.~H.~Smith}

%\authorrunning{Short form of author list} % if too long for running head

\institute{A.~G.~Volosniev  \at
              Institut f{\"u}r Kernphysik, Technische Universit{\"a}t Darmstadt, 64289 Darmstadt, Germany   
  \newline
 \email{volosniev@theorie.ikp.physik.tu-darmstadt.de}    %  \\
%             \emph{Present address:} of F. Author  %  if needed
           \and
           D.~H.~Smith \at Department of Physics, The Ohio State University, Columbus, OH 43210, USA  
}

%\institute{ Department of Physics and Astronomy, Aarhus University, DK-8000 Aarhus C, Denmark \\
 %             \email{artem@phys.au.dk}           %  \\    
%}

\date{Received: date / Accepted: date}
% The correct dates will be entered by the editor



\maketitle

\begin{abstract}
\keywords{one spatial dimension}
\end{abstract}



\section{Introduction}



\section{Scattering off a time-periodic zero-range potential in one dimension}
\label{sec:zero}
\begin{figure}
\centering
\includegraphics[scale=0.7]{fig1new.eps}
\caption{Panel {\textbf a)}: The reflection probability $|B_0|^2$ as a function of $p/\sqrt{\omega}$ for $g_1/\sqrt{\omega}=0.2$ -- black solid curve, $g_1/\sqrt{\omega}=0.4$ -- red dashed curve,  $g_1/\sqrt{\omega}=0.8$ -- green dot-dashed curve, and $g_1/\sqrt{\omega}=1.0$ -- blue dotted curve. For all curves $g_0/\sqrt{\omega}=-1$. Panel {\textbf b)}: $|B_0|^2$ for $g_1/\sqrt{\omega}=1.5$ -- black solid curve, $g_1/\sqrt{\omega}=2.5$ -- red dashed curve, $g_1/\sqrt{\omega}=3.5$ -- green dot-dashed curve, and $g_1/\sqrt{\omega}=4.5$ -- blue dotted curve. For all curves $g_0=0$. Panel {\textbf c)}: $|B_0|^2$ for $g_0/\sqrt{\omega}=0.1$ -- black solid curve, $g_0/\sqrt{\omega}=0.5$ -- red dashed curve,  $g_0/\sqrt{\omega}=1$ -- green dot-dashed curve, and $g_0/\sqrt{\omega}=1.5$ -- blue dotted curve. For all curves $g_1/\sqrt{\omega}=1.5$. Panel {\textbf d)}:  $|B_0|^2$ for $g_0/\sqrt{\omega}=-0.1$ -- black solid curve, $g_0/\sqrt{\omega}=-0.8$ -- red dashed curve,  $g_0/\sqrt{\omega}=-1.4$ -- green dot-dashed curve, and $g_0/\sqrt{\omega}=-2$ -- blue dotted curve. For all curves $g_1/\sqrt{\omega}=1.5$.}
\label{fig:fig1new}
\end{figure}







\section{Conclusions}
\label{sec:concl}

{\small {\bf Acknowledgments} A.~G.~V. gratefully acknowledges the support of the Humboldt Foundation.}

\appendix

\section{Appendix}
\label{sec:formalism}

\subsection{Floquet Formalism}

{\bf Scattering in time-dependent picture}. Here, for convenience of the reader, we discuss the time-dependent Schr{\"o}dinger equation with a general short-range time-periodic potential
\begin{equation}
i\frac{\partial}{\partial t}\Psi(x,t)=H(t)\Psi(x,t),\qquad H(t)=H_0+W(x,t),
\label{eq:schr}
\end{equation}
where the operators in the coordinate representation are $H_0=-\frac{\partial^2}{\partial x^2}$, and $W(x,t)=\sum_n  e^{-\frac{2 i \pi n t}{T}} W_n(x)$. We assume that the incoming square-integrable wave packet is $\Psi_0(x,t)$, i.e., $\Psi(x,t\to-\infty)=\Psi_0(x,t)$. Obviously, if $g_n=0, \forall n$ then the time propagation is $\Psi_0(x,t)=e^{-i H_0 t}\Psi_0$, where $\Psi_0\equiv\Psi_0(x,0)$. Therefore, the effect of scattering is deduced by comparing $\Psi(x,t)$ with $\Psi_0(x,t)$. 

To proceed we note that there is a unitary operator $U(t,s)$ that determines the time evolution
\begin{equation}
\Psi(t)=U(t,s)\Psi(s).
\end{equation}
The formal properties of $U(t,s)$ are discussed in Ref. \cite{yajima1977}.
From now on we reserve the letters $t$ and $s$ for time, also when it does not cause confusion we omit the coordinate variables. For convenience, we set $t=0$ to be a reference time and introduce $\Psi\equiv\Psi(x,0)$ such that $\Psi(t)=U(t)\Psi$, where $U(t)\equiv U(t,0)$. To compare $\Psi(t)$ and $\Psi_0(t)$ one can introduce the wave operator $\Omega(s)$, such that $\Psi=\Omega \Psi_0$, $\Omega=\Omega(0)$. This operator is defined as the limit
\begin{equation}
\Omega(s) \equiv \lim_{t\to-\infty}U^{-1}(t,s)e^{-i(t-s) H_0}.
\end{equation}
In scattering theory for time-independent potentials $\Omega$ is often called the M{\o}ller operator \cite{taylorbook}. This operator exists also for time-periodic short-range potentials as discussed in Refs.~\cite{yajima1977, howland1979}. One important property of $\Omega$ is called the intertwining relation:
\begin{equation}
U(t)\Omega=\Omega(t)e^{-i H_0 t}.
\end{equation}
We use this relation upon decomposing $\Psi_0$ in the eigenbasis of $H_0$. For convenience we refrain from using coordinate representation and use Dirac's notation for vectors instead: $|\Psi_0\rangle=\int_{-\infty}^{\infty} \mathrm{d}p \phi(p) |p\rangle$. Now if we apply the intertwining relation to this decomposition we obtain
\begin{equation}
\int \mathrm{d}p \phi(p)\left(U(t)\Omega |p\rangle -e^{-i p^2 t}\Omega(t)|p\rangle \right)=0.
\end{equation}
Since it should be valid for all possible initial wave packets described by $\phi(p)$ we  conclude that the integrand should be zero. By differentiating both sides with respect to time, we see that the only way to satisfy this condition is to assume that $|f_p\rangle(t)\equiv \Omega(t)|p\rangle$ obeys
\begin{equation}
\left(H(t)-i\frac{\partial}{\partial t}\right) |f_p\rangle(t) =p^2 |f_p\rangle(t).
\label{eq:floquet}
\end{equation}
%To find $\Omega(t)|p\rangle$ initial conditions are needed.

Let us take a closer look at Eq. (\ref{eq:floquet}). According to the Floquet theorem $U(t+T,s+T)=U(t,s)$; see also Ref. \cite{yajima1977}. Therefore, $\Omega(T)=\Omega$, hence $|f_p\rangle(t+T)=|f_p\rangle(t)$ and $|f_p\rangle(t)=\sum_n e^{-\frac{2 i \pi n  t}{T}} |\tilde f_{pn}\rangle$. By inserting this ansatz function into Eq. (\ref{eq:floquet}) and projecting onto a particular mode we obtain 
\begin{equation}
(p^2+n \omega -H_0)| \tilde  f_{pn}\rangle= \sum_m W_{n-m}|\tilde f_{pm}\rangle,
\label{eq:set}
\end{equation}
where $\omega=2\pi/T$. This is a (infinite-dimensional) matrix equation, and therefore, for each $p^2$ there is an infinite number of solutions, which can be formally written as 
\begin{equation}
|\tilde f_{pn}\rangle = |p_n\rangle \alpha_n + \frac{1}{p^2+n\omega-H_0+ i\epsilon}\sum_{m}W_{n-m}|\tilde f_{pm}\rangle,
\label{eq:tildefpn}
\end{equation}
where $\alpha_n$ and $\epsilon\in\mathbb{R}$ are coefficients, and $p_n^2=p^2+n\omega$. We show below that the solution that satisfies the initial condition $\Psi(x,t\to-\infty)\to \Psi_0(x,t)$ has $\alpha_n=\delta_{n,0}$ and $\epsilon>0$. To this end
we write the formal solution to Eq.~(\ref{eq:schr}) 
%we first note that
%\begin{equation}
%|\Psi\rangle\equiv \Omega|\Psi_0\rangle=\int \mathrm{d}p \phi(p) |f_p\rangle,
%\label{eq:psifromf}
%\end{equation}
%where $|f_p\rangle \equiv |f_p\rangle(0)=\sum_n  |\tilde f_{pn}\rangle$. Then 
\begin{equation}
|\Psi\rangle(t)=e^{-i H_0 t}\int_{-\infty}^{\infty}\mathrm{d}p\phi(p)\left(1-i \int_{-\infty}^t \mathrm{d}t' e^{i H_0 t'}W(t')\Omega(t')e^{-i H_0 t'}\right)|p\rangle.
%\; \to \; |\Psi\rangle=\int \mathrm{d}p\phi(p)(|p\rangle + A|p\rangle),
\label{eq:psi_t}
\end{equation}
%where $A=T\mathrm{exp}\left[-i\int_{-\infty}^0\mathrm{d}t e^{i H_0 t} W(t) e^{-i H_0 t}\right]-1$, with $T$ we denote the usual time-ordering operation.
%Another way to write the operator $A$ is 
%\begin{equation}
%A=-i\int_{-\infty}^0 \mathrm{d}t e^{iH_0 t} W(t) \Omega(t) e^{-iH_0 t}.
%\end{equation}
%By acting with $A$ on $|p\rangle$ we deduce the form of $A|p\rangle=|f_p\rangle-|p\rangle$:
%\begin{equation}
%|f_p\rangle -|p\rangle =  - \sum_{m,n}\frac{1}{H_0-p^2+\omega(m+n)-i\epsilon}W_m|%\tilde f_{pn}\rangle,
%\label{eq:form_f}
%\end{equation}
This equation yields $|\Psi\rangle$
\begin{equation}
|\Psi\rangle=\int_{-\infty}^{\infty}\mathrm{d}p \phi(p)\left(|p\rangle+\sum_{m,n}\frac{1}{p^2+(n+m)\omega-H_0+i\delta}W_{n}|\tilde f_{pm}\rangle\right).
\label{eq:psi_help1}
\end{equation} 
where $\delta$ is a small positive quantity. 	
To obtain $|\Psi\rangle$ we used the prescription $W(t)\to e^{\delta t}W(t)$, which is justified by noticing that for $t\to-\infty$ the square-integrable wave packet cannot be affected by the finite-range potential. Now if we look at Eqs.~(\ref{eq:tildefpn}) and~(\ref{eq:psi_help1}) and notice that
$|\Psi\rangle=\Omega |\Psi_0\rangle=\int \mathrm{d}p \phi(p)\sum_n |\tilde f_{pn}\rangle$ we deduce that $\alpha_n=\delta_{n,0}$ and $\epsilon>0$. At $t$ we have $|\Psi\rangle(t)=\int\mathrm{d}p \phi(p)e^{-ip^2t} \sum_{n}e^{-i\omega n t}|\tilde f_{pn}\rangle$. This provides us with the wave function $\Psi(x,t)$ for $x\to\infty$, which determines characteristics of transmission 
\begin{equation}
\Psi(x,t)=\Psi_0(x,t) -\frac{i}{2}\sum_{m,n}\int\mathrm{d}p\phi(p) \frac{e^{ip_nx - i p_n^2 t }}{p_n} \int\mathrm{d}x' e^{-ip_nx'} W_{n-m}(x')\langle x'|\tilde f_{pm}\rangle,
\label{eq:asymptotics}
\end{equation}
a similar expression can be derived for $x\to-\infty$.
Here we use the coordinate representation of the Green's function
\begin{align}
G(x,x';k^2)\equiv\langle x| \frac{1}{H_0-k^2-i\epsilon}|x'\rangle  = \frac{i}{2k}e^{ik|x-x'|}.
\end{align}


{\bf Scattering in time-independent picture}. In this subsection we consider Eq.~(\ref{eq:tildefpn}) that determines the properties of the scattering in time-independent picture in more detail. In the coordinate representation it reads
\begin{align}
\tilde f_{pn}(x)=\delta_{n,0}e^{i p_n x}- \sum_{m=-\infty}^{\infty} \int\mathrm{d}x' 
 G(x,x';p_n^2)W_{n-m}(x')\tilde f_{pm}(x'),
\label{eq:fpn}
\end{align} 
where $\delta_{m,n}$ is Kronecker's delta. This function at $x\to\infty$ has the form $\delta_{n,0}e^{ip_nx}+B_ne^{ip_nx}$ where 
\begin{align}
B_n=-\frac{i}{2 p_n}\sum_{m=-\infty}^\infty \int\mathrm{d}x' e^{-i p_n x'}W_{n-m}(x')\tilde f_{pm}(x')
\label{eq:an}
\end{align}
whereas at $x\to-\infty$ it has the form $\delta_{n,0}e^{ip_nx}+\tilde B_n e^{-i p_n x}$ with
\begin{align}
\tilde B_n=-\frac{i}{2p_n}\sum_{m=-\infty}^\infty \int\mathrm{d}x' e^{i p_n x'}W_{n-m}(x')\tilde f_{pm}(x'), 
\label{eq:bn}
\end{align}
Apparently, Eqs. (\ref{eq:fpn}), (\ref{eq:an}) and (\ref{eq:bn}) contain all information about the scattering process and can be used to derive Eq.~(\ref{eq:set_zero}) of the main text. It is worthwhile to notice that for a plane wave with a given $p^2$ the total probability to find a particle with the energies $p^2_n, n=0,\pm1,...$ is conserved in the scattering process (see the next subsection). This can be seen as the conservation of the quasi-energy. At the same time the total energy is not conserved and can become larger or smaller, depending on the problem.

\subsection{Conservation of Flux.}
\label{sec:appA}

The Floquet modes, $f_p$, fully describe the scattering process. Since they always contain scattering states, there should be no probability to find a particle close to the potential at $t\to\infty$ (assuming a square-integrable wave function at $t\to-\infty$): the total outgoing flux should be equal to the total incoming flux. Note that since the states with $p^2-\omega(m+n)<0$ do not give any contribution to the fluxes, the particle will leave these modes after some time. Physically it is easily understood, since a particle in these modes can undergo a transition to a scattering state and leave the range of the potential. This appendix 
shows the conservation of flux explicitly in a time-independent picture. Let us start with the scattering off zero-range potential described by Eq.~(\ref{eq:set_zero}), for which the flux is 
\begin{equation}
\vec j = i (\psi \vec \nabla \psi^*-\psi^* \vec \nabla \psi).
\end{equation}
The incoming flux is along the $x$ axis and amounts to $2p$.
The outgoing flux consists of the two parts: the first is along the $x$ direction and equals to $\sum_{p_n\geq0} 2p_n|C_n|^2$. The second piece is along the $(-x)$
 direction and amounts to $\sum_{p_n\geq0} 2 p_n |C_n-\delta_{n,0}|^2$. Let us show that from Eq.~(\ref{eq:set_zero}) it follows that 
\begin{equation}
\sum_{p_n\geq 0}p_n|C_n|^2=p \mathrm{Re} C_0,
\label{eq:fluxzerorange}
\end{equation}
which means that the total flux is conserved. To this end, we multiply
Eq.~(\ref{eq:set_zero}) with~$C_n^{*}$,
\begin{equation}
2 p_n |C_n|^2=2p\delta_{n,0} C_n^*-ig_0 |C_n|^2 - \frac{i g_1}{2}(C_{n+1}+C_{n-1})C_n^*.
\end{equation}
Next we conjugate Eq.~(\ref{eq:set_zero}) and then multiply
with~$C_n$,
\begin{equation}
2 p_n^* |C_n|^2=2p\delta_{n,0} C_n+ig_0 |C_n|^2 + \frac{i g_1}{2}(C^*_{n+1}+C^*_{n-1})C_n.
\end{equation}
Now we add these two equations and sum over all states
\begin{equation}
\sum (p_n+p_n^*)|C_n|^2=2p \mathrm{Re} C_0.
\end{equation}
Since $p_n+p_n^*$ is non-zero only for the scattering states we obtain Eq. (\ref{eq:fluxzerorange}). 
%Note, that if we instead sum up over the scattering states we obtain 
%\begin{equation}
%\sum_{p_n\geq 0} p_n|C_n|^2=2 p \mathrm{Re} C_0 +\frac{i g_1}{4}(C_l C_{l-1}^*-C_l^* C_{l-1}),
%\end{equation}
%where $l$ corresponds to the lowest in energy scattering wave, e.g., if $p^2<\omega$ then $l=0$. This establishes the following identity $\mathrm{Im} (C_{l}^*C_{l-1})=0$. From this identity one can easily derive that $\mathrm{Im} (C_{l-k}^*C_{l-n-1})=0$ for any $n,k\geq 0$. 
Similar steps can be taken to show that the total flux is conserved for any short-range potential. 
%For this we need to show that 
%\begin{equation}
%\sum_{p_n\geq0}p_n(|A_n|^2+|B_n^2|)=-2p\mathrm{Re}A_0,
%\label{eq:appa_cond}
%\end{equation}
%where $A_n$ and $B_n$ are taken from Eqs. (\ref{eq:an}) and (\ref{eq:bn}) respectively.
%Using the definitions of $A_n$ and $B_n$ this expression can be written as
%\begin{align}
%\mathrm{lhs}=-i\sum_{p_n\geq 0,m',m''} \int \mathrm{d}x'\mathrm{d}x'' W_{n-m'}(x')W_{n-m''}(x'') \left[ G(x'',x';p_n^2)-G^*(x'',x';p_n^2)\right]f_{pm'}(x')f^*_{pm''}(x'').
%\end{align}
%To derive Eq. (\ref{eq:appa_cond}) one should use Eq.~(\ref{eq:fpn}) having in mind that for every $n$, such that $p_n^2<0$, the following holds 
%\begin{equation}
%\sum_m \int \mathrm{d}x W_{n-m}(x)\left[f_{pm}^*(x)f_{pn}(x)-f_{pm}(x)f_{pn}^*(x)\right]=0.
%\end{equation}
%which generalizes the previously derived relation $\mathrm{Im}(C_l^*C_{l-1})=0$. 
%This equation can be obtained directly from Eq. (\ref{eq:set}).





\bibliographystyle{unsrt}
\bibliography{bib}

\end{document}
{\bf Appendix -- Limiting cases}

\section{Energy transfer}
\label{sec:transfer}
